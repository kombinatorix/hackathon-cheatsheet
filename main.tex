


\documentclass{article}
\usepackage[landscape]{geometry}
\usepackage{url}
\usepackage{multicol}
\usepackage{amsmath}
\usepackage{amsfonts}
\usepackage{tikz}
\usetikzlibrary{decorations.pathmorphing}
\usepackage{amsmath,amssymb}

\usepackage{colortbl}
\usepackage{xcolor}
\usepackage{mathtools}
\usepackage{amsmath,amssymb}
\usepackage{enumitem}

%For correct tilde
\usepackage{times}
\usepackage{textcomp}
% \texttildelow

\title{Hackathon Cheatsheet}
\usepackage[english]{babel}
\usepackage[utf8]{inputenc}

\advance\topmargin-.8in
\advance\textheight3in
\advance\textwidth3in
\advance\oddsidemargin-1.5in
\advance\evensidemargin-1.5in
\parindent0pt
\parskip2pt
\newcommand{\hr}{\centerline{\rule{3.5in}{1pt}}}
%\colorbox[HTML]{e4e4e4}{\makebox[\textwidth-2\fboxsep][l]{texto}
\begin{document}

\begin{center}{\huge{\textbf{Hackathon Cheatsheet}}}\\

\end{center}
\begin{multicols*}{3}

\tikzstyle{mybox} = [draw=black, fill=white, very thick,
    rectangle, rounded corners, inner sep=10pt, inner ysep=10pt]
\tikzstyle{fancytitle} =[fill=black, text=white, font=\bfseries]

\begin{tikzpicture}
\node [mybox] (box){%
    \begin{minipage}{0.3\textwidth}
		Generate your own ssh-key and set permissions: \\
        \\
		\textbf{\$ ssh-keygen -t rsa} \\
		\textbf{\$ chmod 700 \texttildelow/.ssh} \\
		\textbf{\$ chmod 600 \texttildelow/.ssh/id\_rsa} \\
		\textbf{\$ chmod 644 \texttildelow/.ssh/id\_rsa.pub} \\
        \\
        Log into the Hackathon GitLab with your given credentials.\\
        \begin{enumerate}
        \item Go to \textbf{settings} (Icon on the upper right).
        \item Click on \textbf{SSH Keys}.
        \item \textbf{\$ cat \texttildelow/.ssh/id\_rs.pub}
        \item Copy everything to the \textbf{Key} textarea and save.
        \end{enumerate}
    \end{minipage}
};

\node[fancytitle, right=10pt] at (box.north west) {GitLab Basics};
\end{tikzpicture}



\begin{tikzpicture}
\node [mybox] (box){%
    \begin{minipage}{0.3\textwidth}
		Configure your Git: \\
        \\
		\textbf{\$ git config -\,-global user.name \;"Your Name"} \\
		\textbf{\$ git config -\,-global user.email \;"you@example.com"} \\
		\textbf{\$ git config -\,-global color.ui auto} \\
		\\
    \end{minipage}
};

\node[fancytitle, right=10pt] at (box.north west) {Git configuration};
\end{tikzpicture}

\begin{tikzpicture}
\node [mybox] (box){%
    \begin{minipage}{0.3\textwidth}
    	Enter the right directory:\\
    	\textbf{\$ cd fawkes-robotino/fawkes/...}\\
        Clone the right repository:\\
        \textbf{\$ git clone ...}\\
        Checkout your branch:\\
        \textbf{\$ git checkout team-\textit{*your number*}}\\
        \\
        Start hacking and read about the git-workflow.
    \end{minipage}
};

\node[fancytitle, right=10pt] at (box.north west) {Get the code};
\end{tikzpicture}

\begin{tikzpicture}
\node [mybox] (box){%
    \begin{minipage}{0.3\textwidth}
		You made your first changes and want to save your progress. You can display all files changed by entering:\\
		\textbf{\$ git status}\\
		Group all files that were changed and contribute to a feature an add them:\\
		\textbf{\$ git add [files]}\\
		Commit the added files. Write a helpful commit message. It will save you time. \\ 
		\textbf{\$ git commit } will open an editor.\\
		\textbf{\$ git commit -m "commit message"} will take the short commit message.\\
		Now you should push that:\\
		\textbf{\$ git push}
    \end{minipage}
};

\node[fancytitle, right=10pt] at (box.north west) {Git workflow};
\end{tikzpicture}

\begin{tikzpicture}
\node [mybox] (box){%
    \begin{minipage}{0.3\textwidth}
    ... you added the wrong file:\\
    \textbf{\$ git reset [file]}\\ \\
	... \textbf{git commit} opened the wrong editor:\\
	\textbf{\$ git config --global core.editor "vim"}\\ or "emacs" if you like.\\ \\
	... don't know what you've changed:\\
	\textbf{\$ git diff [file]}\\
	You can look up more options at \url{https://git-scm.com/docs/git-diff}\\ \\
	... you search for a specific commit:\\
	\textbf{\$ git log}\\
	or do so interactively:\\
	\textbf{\$ gitk}\\ \\
	... you want to revert a specific commit:\\
	\textbf{\$ git revert [hash-of-commit]}
    \end{minipage}
};

\node[fancytitle, right=10pt] at (box.north west) {What to do, if};
\end{tikzpicture}

\begin{tikzpicture}
\node [mybox] (box){%
    \begin{minipage}{0.3\textwidth}
	\textbf{\$ git add [files]}\\
	If the fire is near:\\
	\textbf{\$ git add .}\\
	After that:\\
	\textbf{\$ git commit -m "fire"}\\
	\textbf{\$ git push}\\ \\
	Run!
    \end{minipage}
};

\node[fancytitle, right=10pt] at (box.north west) {In case of fire};
\end{tikzpicture}


\begin{tikzpicture}
\node [mybox] (box){%
	\begin{minipage}{0.3\textwidth}
	Maybe you are new to Linux and want to go to different directories, you should use \textbf{cd}:\\
	\textbf{\$ cd \textit{path}}\\
	If you want to go to your home-directory:\\
	\textbf{\$ cd}\\
	If you qant to go somewhere relative to your home-directory:\\
	\textbf{\$ cd \texttildelow/\textit{path}}\\
	If you want to go a directory up:\\
	\textbf{\$ cd ..}\\ \\
	
	If you want to know what files and subdirectories are in the current directory:\\
	\textbf{\$ ls}\\ \\
		
	If you think you need inspiration on your code, you can use grep to get your inspiration:\\
	\textbf{\$ grep [options] \textit{searchstring} [files]}\\
	If you don't know which file could possible contain your search string, grep can recursively go through the directories, starting from the current one:\\
	\textbf{\$ grep -R \textit{searchstring}}
	\end{minipage}
};

\node[fancytitle, right=10pt] at (box.north west) {Usefull tips};
\end{tikzpicture}

\begin{tikzpicture}
\node [mybox] (box){%
	\begin{minipage}{0.3\textwidth}
	Change your directory to the following:\\
	\textbf{\$ cd \texttildelow/fawkes-robotino/bin}\\
	Start the simulation with:\\
	\textbf{\$ ./gazsim.bash -x start -n 1 -r -a}
	\end{minipage}
};

\node[fancytitle, right=10pt] at (box.north west) {Get the Simulation running};
\end{tikzpicture}
\end{multicols*}
\newpage
\begin{center}{\huge{\textbf{Hackathon Cheatsheet - CLIPS}}}\\

\end{center}
\begin{multicols*}{3}
	
	\tikzstyle{mybox} = [draw=black, fill=white, very thick,
	rectangle, rounded corners, inner sep=10pt, inner ysep=10pt]
	\tikzstyle{fancytitle} =[fill=black, text=white, font=\bfseries]

	\begin{tikzpicture}
	\node [mybox] (box){%
		\begin{minipage}{0.3\textwidth}
		The  Language Integrated Production System (\textbf{CLIPS}) consists mainly of the following components:\\
		\begin{center}
		\begin{tabular}{lp{5.5cm} l}
		Facts & Representation of knowledge\\
		\hline Templates & Abstract structure for facts\\
		\hline Variables & Variables that only life within the scope they are defined in\\
		\hline Globals & Variables with a global scope\\
		\hline Functions & Sequence of actions. Similiar to other programming languages \\
		\hline Rules & The heart of CLIPS. If-Then-structure which fires if the lefthandside (LHS) is fulfilled.\\ 
		\end{tabular}
		\end{center}\end{minipage}
	};

	\node[fancytitle, right=10pt] at (box.north west) {CLIPS overview};
	\end{tikzpicture}
	
	

	\begin{tikzpicture}
	\node [mybox] (box){%
		\begin{minipage}{0.3\textwidth}
		A fact consists of a \textbf{factname} and its \textbf{content}. There can usally only be one unique fact.
		You can add a fact as following:\\
		\textbf{(assert (factname content1 content2 ... contentN))}\\ \\
		You can also have a \textbf{deffacts} construct, which asserts facts automatically wheneever the \textbf{reset} command is performed. It consits of a unique name an optional comment and at least one fact:\\

		(deffacts deffactsname "optional comment"\\
		$\-$\hspace{5mm} (fact1 content1 content2)\\
		$\-$\hspace{5mm} (fact2 content3)\\
		)
		\end{minipage}
	};

	\node[fancytitle, right=10pt] at (box.north west) {Facts};
	\end{tikzpicture}

	\begin{tikzpicture}
	\node [mybox] (box){%
		\begin{minipage}{0.3\textwidth}
		A template consists of a \textbf{templatename} different slots and multislots, which can be constrained.
		You can construct a template as follows:\\ \\
		(deftemplate templatename "optional comment"\\
		$\-$\hspace{5mm}(slot slot-name1 (default ?NONE))\\
		$\-$\hspace{5mm}(slot slot-name2 (type INTEGER))\\
		$\-$\hspace{5mm}(multislot slot-name3 (allowed-values 1 3.5 "A"))\\
		)\\
		A fact can be asserted like this:\\
		\textbf{(assert (templatename\\
			(slotname1 "foo")  (slot-name2 2)(slot-name3 3.5 1 1)))}
		\end{minipage}
	};

	\node[fancytitle, right=10pt] at (box.north west) {Templates 1};
	\end{tikzpicture}

	\begin{tikzpicture}
	\node [mybox] (box){%
		\begin{minipage}{0.3\textwidth}
		A slot can have a default value. One speciality ist \textbf{\$NONE}. This means this slot must be filled when a fact for this template is asserted.\\ \\
		Additionaly, it can be specified which type values of this slot must have. Allowed types are: \textbf{SYMBOL, STRING, NUMBER, INTEGER,} and\textbf{ FLOAT.}\\
		One can also specify which values are allowed. This can be archieved by \textbf{allowed-} and \textbf{symbols, strings, numbers, integers, floats,} and\textbf{ values.}\\
		For numbers the allowed values can be constraint by a range, for example: \textbf{(range 1 375)}\\ \\
		In multifield-slots the number of stored values can be restricted, for example: \textbf{(cardinality 17)}
		\end{minipage}
	};

	\node[fancytitle, right=10pt] at (box.north west) {Templates 2};
	\end{tikzpicture}

	\begin{tikzpicture}
	\node [mybox] (box){%
		\begin{minipage}{0.3\textwidth}
		A variable can be defined like this:\\
		\textbf{(bind ?variable-name 1)}\\
		A multifield-variable can be defined:\\
		\textbf{(bind ?variable-name (create\$ 1 1 2 3 5 8))}\\
		One can get the length of a multifield-variable with:\\
		\textbf{(length ?multifield-variable)}\\
		Moreover, one can extract the $n$-th element. For example the second element:\\
		\textbf{(nth\$ 2 ?multifield-variable)}
		\end{minipage}
	};

	\node[fancytitle, right=10pt] at (box.north west) {Variables};
	\end{tikzpicture}

	\begin{tikzpicture}
	\node [mybox] (box){%
		\begin{minipage}{0.3\textwidth}
		Global variables are similar to variables, except that they have a global scope and they have a different syntax:\\
		\textbf{?*foo*} is a global variable, but \textbf{?foo} is not. Globals can be defined within a \textbf{defglobal} block:\\
		(defglobal \\
		$\-$\hspace{5mm}?*foo* = 1\\
		$\-$\hspace{5mm}?*bar* = "FooBar"\\
		)
		\end{minipage}
	};

	\node[fancytitle, right=10pt] at (box.north west) {Globals};
	\end{tikzpicture}

	\begin{tikzpicture}
	\node [mybox] (box){%
		\begin{minipage}{0.3\textwidth}
		Functions are similiar to other languages. An example:\\
		(deffunction add-x-to-y (?y ?x)\\
		$\-$\hspace{5mm}(bind ?y (+ ?y ?x))\\
		$\-$\hspace{5mm}(return ?y)\\
		)\\
		A return value is not needed.
		\end{minipage}
	};

	\node[fancytitle, right=10pt] at (box.north west) {Functions};
	\end{tikzpicture}
	

	\begin{tikzpicture}
	\node [mybox] (box){%
		\begin{minipage}{0.3\textwidth}
		Rules are the heart of CLIPS. A rule consists maily of a name, a LHS, and a RHS.
		If, after a new fact was asserted, the LHS of rule matches, the RHS will be executed.
		An simple example would be if A is true, then B true follows:\\ \\
		(defrule a-follows-b\\
		$\-$\hspace{5mm}(A true)\\
		=$>$\\
		$\-$\hspace{5mm}(assert (B true))\\
		)\\
		But maybe we want that, when A is true or false and B is false, B should be deleted and added as true:\\ \\
		(defrule a-will-be-deleted\\
		$\-$\hspace{5mm}(A ?whatever)\\
		$\-$\hspace{5mm}?b $<$- (B false)\\
		=$>$\\
		$\-$\hspace{5mm}(retract ?b)\\
		$\-$\hspace{5mm}(assert (B true))\\
		)\\
		Another scenario would be that, if a counter is above a certain threshold, A should be deleted and the counter set to 0:\\
		(defglobal ?*counter-threshold*)\\
		(deftemplate counter (slot steps (type INTEGER) ) )\\
		(defrule countermagic\\
		$\-$\hspace{5mm}?counter (counter (steps ?steps\&:($>$5 ?steps ?*counter-threshold*)\\
		$\-$\hspace{5mm}?a $<$- (A ?whatever)\\
		=$>$\\
		$\-$\hspace{5mm}(retract ?a)\\
		$\-$\hspace{5mm}(bind ?steps 0)\\
		$\-$\hspace{5mm}(modify ?m (steps ?steps))\\
		)
		\end{minipage}
	};

	\node[fancytitle, right=10pt] at (box.north west) {Rules};
	\end{tikzpicture}

	\begin{tikzpicture}
	\node [mybox] (box){%
		\begin{minipage}{0.3\textwidth}
		More examples:\\
		\url{https://github.com/kombinatorix/clips-examples}
		
		\end{minipage}
	};

	\node[fancytitle, right=10pt] at (box.north west) {Links};
	\end{tikzpicture}
\end{multicols*}
\end{document}